\documentclass[12pt,a4paper]{article}

% ===== PACKAGES =====
\usepackage{fontspec}
\usepackage{polyglossia}
\setmainlanguage{vietnamese}
\setmainfont{Times New Roman}
\setmonofont{Menlo}
\usepackage{geometry}
\usepackage{amssymb}
\usepackage{graphicx}
\usepackage{tikz}
\usepackage{xcolor}
\usepackage{listings}
\usepackage{hyperref}
\usepackage{fancyhdr}
\usepackage{titlesec}
\usepackage{booktabs}
\usepackage{array}
\usepackage{multirow}
\usepackage{enumitem}
\usepackage{tcolorbox}
\usepackage{fontawesome5}

% TikZ Libraries
\usetikzlibrary{shapes.geometric, arrows.meta, positioning, fit, backgrounds, calc, shadows, decorations.pathmorphing}

% Circular image command
\newcommand{\circleimage}[2]{%
    \begin{tikzpicture}
        \clip (0,0) circle (#1);
        \node at (0,0) {\includegraphics[width=#1*2]{#2}};
    \end{tikzpicture}%
}

% ===== PAGE SETUP =====
\geometry{margin=2.5cm}
\pagestyle{fancy}
\fancyhf{}
\fancyhead[L]{\textbf{LexiLingo}}
\fancyhead[R]{Mô Tả Kiến Trúc Phần Mềm}
\fancyfoot[C]{\thepage}

% ===== COLORS =====
\definecolor{primaryblue}{RGB}{66, 133, 244}
\definecolor{secondarygreen}{RGB}{52, 168, 83}
\definecolor{accentorange}{RGB}{251, 188, 4}
\definecolor{alertred}{RGB}{234, 67, 53}
\definecolor{darkgray}{RGB}{95, 99, 104}
\definecolor{lightgray}{RGB}{241, 243, 244}
\definecolor{codebg}{RGB}{248, 249, 250}

% ===== TITLE FORMATTING =====
\titleformat{\section}{\Large\bfseries\color{primaryblue}}{\thesection}{1em}{}
\titleformat{\subsection}{\large\bfseries\color{darkgray}}{\thesubsection}{1em}{}

% ===== CUSTOM BOXES =====
\newtcolorbox{infobox}[1][]{
    colback=lightgray,
    colframe=primaryblue,
    fonttitle=\bfseries,
    title=#1,
    rounded corners
}

\newtcolorbox{featurebox}{
    colback=white,
    colframe=secondarygreen,
    rounded corners,
    boxrule=1pt
}

% ===== DOCUMENT =====
\begin{document}

% ===== TITLE PAGE =====
\begin{titlepage}
    \centering
    \vspace*{2cm}
    
    {\Huge\bfseries\color{primaryblue} LexiLingo}\\[0.5cm]
    {\Large\color{darkgray} AI-Powered English Learning Application}\\[2cm]
    
    % Logo (already circular with transparent background)
    \includegraphics[width=5cm]{logo.png}\\[1cm]
    
    {\LARGE\bfseries MÔ TẢ KIẾN TRÚC}\\[0.5cm]
    {\large Mobile Application Architecture}\\[2cm]
    
    \begin{tabular}{ll}
        \textbf{Phiên bản:} & v1.2.7 \\
        \textbf{Ngày:} & 14/01/2026 \\
        \textbf{Nền tảng:} & Flutter (iOS/Android/Web) \\
        \textbf{Trạng thái:} & Development \\
    \end{tabular}
    
    \vfill
    {\large\color{darkgray} Nguyen Huu Thang - Lead Developer}\\
    {\large\color{darkgray} Email:nhthang312@gmail.com}
\end{titlepage}

% ===== TABLE OF CONTENTS =====
\tableofcontents
\newpage

% ===== SECTION 1: TỔNG QUAN =====
\section{Tổng Quan Dự Án}

\subsection{Giới Thiệu}

\textbf{LexiLingo} là ứng dụng học tiếng Anh thông minh sử dụng AI để hỗ trợ người học cải thiện kỹ năng ngôn ngữ một cách hiệu quả. Ứng dụng tập trung vào việc cung cấp trải nghiệm học tập cá nhân hóa thông qua:

\begin{itemize}[leftmargin=2cm]
    \item Hội thoại với AI Tutor
    \item Học từ vựng theo ngữ cảnh
    \item Kiểm tra ngữ pháp tự động
    \item Luyện phát âm với phản hồi thời gian thực
    \item Theo dõi tiến độ học tập
\end{itemize}

\subsection{Công Nghệ Sử Dụng}

\begin{table}[h]
\centering
\begin{tabular}{|l|l|l|}
\hline
\textbf{Layer} & \textbf{Công Nghệ} & \textbf{Mục Đích} \\
\hline
Frontend & Flutter 3.29+ & Cross-platform UI \\
State Management & Provider & Quản lý trạng thái \\
Database & SQLite + Firestore & Local \& Cloud storage \\
AI/ML & Qwen2.5 + Whisper + HuBERT & Language processing \\
Authentication & Firebase Auth & Xác thực người dùng \\
\hline
\end{tabular}
\caption{Stack công nghệ chính}
\end{table}

% ===== SECTION 2: KIẾN TRÚC TỔNG QUAN =====
\section{Kiến Trúc Tổng Quan}

\subsection{Clean Architecture}

LexiLingo được xây dựng theo mô hình \textbf{Clean Architecture} kết hợp với \textbf{Feature-First Structure}, đảm bảo:

\begin{itemize}
    \item \textbf{Separation of Concerns}: Tách biệt rõ ràng các tầng logic
    \item \textbf{Testability}: Dễ dàng viết unit test
    \item \textbf{Maintainability}: Bảo trì và mở rộng thuận tiện
    \item \textbf{Scalability}: Có thể scale khi cần thiết
\end{itemize}

\subsection{Sơ Đồ Kiến Trúc Tổng Quan}

\begin{center}
\begin{tikzpicture}[
    node distance=1.2cm,
    layer/.style={
        rectangle,
        rounded corners=5pt,
        minimum width=12cm,
        minimum height=1.5cm,
        text centered,
        font=\bfseries
    },
    arrow/.style={
        -Stealth,
        thick
    }
]

% Layers
\node[layer, fill=primaryblue!20, draw=primaryblue, line width=1.5pt] (presentation) 
    {\color{primaryblue}\faDesktop\ PRESENTATION LAYER};

\node[layer, fill=secondarygreen!20, draw=secondarygreen, line width=1.5pt, below=of presentation] (domain) 
    {\color{secondarygreen}\faCogs\ DOMAIN LAYER};

\node[layer, fill=accentorange!20, draw=accentorange, line width=1.5pt, below=of domain] (data) 
    {\color{accentorange}\faDatabase\ DATA LAYER};

\node[layer, fill=alertred!20, draw=alertred, line width=1.5pt, below=of data] (external) 
    {\color{alertred}\faCloud\ EXTERNAL SERVICES};

% Arrows
\draw[arrow, primaryblue] (presentation) -- (domain);
\draw[arrow, secondarygreen] (domain) -- (data);
\draw[arrow, accentorange] (data) -- (external);

% Labels on right
\node[right=0.5cm of presentation, text=darkgray, font=\small] {Widgets, Providers, Screens};
\node[right=0.5cm of domain, text=darkgray, font=\small] {Entities, UseCases, Repositories};
\node[right=0.5cm of data, text=darkgray, font=\small] {DataSources, Models, Impl};
\node[right=0.5cm of external, text=darkgray, font=\small] {Firebase, Qwen2.5, Whisper, SQLite};

\end{tikzpicture}
\end{center}

\newpage

% ===== SECTION 3: CHI TIẾT KIẾN TRÚC =====
\section{Chi Tiết Kiến Trúc Các Layer}

\subsection{Presentation Layer}

Tầng này chịu trách nhiệm về giao diện người dùng và quản lý trạng thái UI.

\begin{center}
\begin{tikzpicture}[
    node distance=0.8cm,
    box/.style={
        rectangle,
        rounded corners=3pt,
        minimum width=3.5cm,
        minimum height=1cm,
        text centered,
        draw=primaryblue,
        fill=primaryblue!10,
        font=\small
    }
]

% Widgets
\node[box] (screens) {Screens};
\node[box, right=of screens] (widgets) {Widgets};
\node[box, right=of widgets] (providers) {Providers};

% Container
\node[draw=primaryblue, dashed, rounded corners, fit=(screens)(widgets)(providers), 
      inner sep=10pt, label={[font=\bfseries\color{primaryblue}]above:Presentation Layer}] {};

\end{tikzpicture}
\end{center}

\textbf{Thành phần chính:}
\begin{itemize}
    \item \textbf{Screens}: Các màn hình chính (ChatScreen, HomeScreen, CourseScreen...)
    \item \textbf{Widgets}: UI components tái sử dụng (MessageBubble, CourseCard...)
    \item \textbf{Providers}: State management với ChangeNotifier pattern
\end{itemize}

\subsection{Domain Layer}

Tầng nghiệp vụ chứa business logic thuần túy, không phụ thuộc vào framework.

\begin{center}
\begin{tikzpicture}[
    node distance=0.8cm,
    box/.style={
        rectangle,
        rounded corners=3pt,
        minimum width=3.5cm,
        minimum height=1cm,
        text centered,
        draw=secondarygreen,
        fill=secondarygreen!10,
        font=\small
    }
]

\node[box] (entities) {Entities};
\node[box, right=of entities] (usecases) {UseCases};
\node[box, right=of usecases] (repos) {Repository (Abstract)};

\node[draw=secondarygreen, dashed, rounded corners, fit=(entities)(usecases)(repos), 
      inner sep=10pt, label={[font=\bfseries\color{secondarygreen}]above:Domain Layer}] {};

\end{tikzpicture}
\end{center}

\textbf{Thành phần chính:}
\begin{itemize}
    \item \textbf{Entities}: Business objects (ChatMessage, User, Course, Vocabulary)
    \item \textbf{UseCases}: Các use case cụ thể (SendMessageUseCase, GetCoursesUseCase)
    \item \textbf{Repository}: Interface định nghĩa contract cho data layer
\end{itemize}

\subsection{Data Layer}

Tầng dữ liệu xử lý việc lấy và lưu trữ dữ liệu từ các nguồn khác nhau.

\begin{center}
\begin{tikzpicture}[
    node distance=0.8cm,
    box/.style={
        rectangle,
        rounded corners=3pt,
        minimum width=3cm,
        minimum height=1cm,
        text centered,
        draw=accentorange,
        fill=accentorange!10,
        font=\small
    }
]

\node[box] (models) {Models};
\node[box, right=of models] (datasources) {DataSources};
\node[box, right=of datasources] (repoimpl) {Repository Impl};

\node[draw=accentorange, dashed, rounded corners, fit=(models)(datasources)(repoimpl), 
      inner sep=10pt, label={[font=\bfseries\color{accentorange}]above:Data Layer}] {};

\end{tikzpicture}
\end{center}

\textbf{Thành phần chính:}
\begin{itemize}
    \item \textbf{Models}: Data Transfer Objects với JSON serialization
    \item \textbf{DataSources}: Local (SQLite) và Remote (API, Firebase)
    \item \textbf{Repository Impl}: Triển khai cụ thể của Repository interface
\end{itemize}

\newpage

% ===== SECTION 4: MODULE STRUCTURE =====
\section{Cấu Trúc Module (Feature-First)}

\subsection{Tổ Chức Thư Mục}

\begin{tcolorbox}[colback=codebg, colframe=darkgray, title=Project Structure]
\begin{verbatim}
lib/
├── core/                    # Shared utilities
│   ├── di/                  # Dependency Injection
│   ├── services/            # Core services
│   ├── theme/               # App theming
│   └── utils/               # Utilities
│
├── features/                # Feature modules
│   ├── auth/                # Authentication
│   ├── chat/                # AI Chat (NEW)
│   ├── course/              # Courses
│   ├── home/                # Home dashboard
│   ├── user/                # User profile
│   └── vocabulary/          # Vocabulary
│
└── main.dart                # Entry point
\end{verbatim}
\end{tcolorbox}

\subsection{Cấu Trúc Một Feature Module}

Mỗi feature module được tổ chức theo 3 layer:

\begin{center}
\begin{tikzpicture}[
    node distance=0.5cm,
    folder/.style={
        rectangle,
        rounded corners=2pt,
        minimum width=3.5cm,
        minimum height=0.8cm,
        text centered,
        font=\small\ttfamily
    },
    subfolder/.style={
        rectangle,
        rounded corners=2pt,
        minimum width=3cm,
        minimum height=0.7cm,
        text centered,
        font=\scriptsize\ttfamily,
        fill=gray!10,
        draw=gray
    }
]

% Feature container
\node[folder, fill=primaryblue!20, draw=primaryblue] (feature) at (0,0) {feature/chat/};

% Subdirectories - increased horizontal spacing
\node[folder, fill=primaryblue!10, draw=primaryblue!50, below left=1.2cm and 2.5cm of feature] (presentation) {presentation/};
\node[folder, fill=secondarygreen!10, draw=secondarygreen!50, below=1.2cm of feature] (domain) {domain/};
\node[folder, fill=accentorange!10, draw=accentorange!50, below right=1.2cm and 2.5cm of feature] (data) {data/};

% Sub-subdirectories - increased vertical spacing
\node[subfolder, below=0.8cm of presentation] (prov) {providers/};
\node[subfolder, below=0.5cm of prov] (screens) {screens/};
\node[subfolder, below=0.5cm of screens] (widgets) {widgets/};

\node[subfolder, below=0.8cm of domain] (entities) {entities/};
\node[subfolder, below=0.5cm of entities] (usecases) {usecases/};
\node[subfolder, below=0.5cm of usecases] (repos) {repositories/};

\node[subfolder, below=0.8cm of data] (datasrc) {datasources/};
\node[subfolder, below=0.5cm of datasrc] (models) {models/};
\node[subfolder, below=0.5cm of models] (repoimpl) {repositories/};

% Lines
\draw[gray] (feature) -- (presentation);
\draw[gray] (feature) -- (domain);
\draw[gray] (feature) -- (data);

\end{tikzpicture}
\end{center}

\newpage

% ===== SECTION 5: AI CHAT MODULE =====
\section{Module AI Chat}

\subsection{Tổng Quan Module}

Module \textbf{Chat} là tính năng cốt lõi cho phép người dùng tương tác với AI Tutor để học tiếng Anh.

\begin{featurebox}
\textbf{Tính năng chính:}
\begin{itemize}
    \item Hội thoại với AI tutor bằng văn bản
    \item Kiểm tra và sửa lỗi ngữ pháp tự động
    \item Giải thích từ vựng theo ngữ cảnh
    \item Đánh giá độ trôi chảy (fluency scoring)
    \item Lưu trữ lịch sử hội thoại
\end{itemize}
\end{featurebox}

\subsection{Kiến Trúc AI Chat Module}

\begin{center}
\begin{tikzpicture}[node distance=0.8cm, scale=0.75]
    % UI Layer
    \node[rectangle, rounded corners, fill=primaryblue!30, draw=primaryblue, minimum width=2.5cm, minimum height=0.8cm] (screen) {ChatScreen};
    \node[rectangle, rounded corners, fill=primaryblue!20, draw=primaryblue, minimum width=2.5cm, minimum height=0.8cm, right=1cm of screen] (provider) {ChatProvider};
    
    % Domain Layer
    \node[rectangle, rounded corners, fill=secondarygreen!30, draw=secondarygreen, minimum width=2.5cm, minimum height=0.8cm, below=0.8cm of screen] (usecase) {UseCase};
    
    % Data Layer
    \node[rectangle, rounded corners, fill=accentorange!30, draw=accentorange, minimum width=2.5cm, minimum height=0.8cm, below=0.8cm of usecase] (repo) {Repository};
    
    % Arrows
    \draw[-Stealth, thick] (screen) -- (provider);
    \draw[-Stealth, thick] (screen) -- (usecase);
    \draw[-Stealth, thick] (usecase) -- (repo);
\end{tikzpicture}
\end{center}

\textbf{Thành phần:} ChatScreen (UI) $\to$ ChatProvider (State) $\to$ UseCase (Logic) $\to$ Repository (Data)

\subsection{AI Service Integration}

Module Chat tích hợp với 3 AI services chính:

\begin{center}
\begin{tikzpicture}[node distance=0.6cm, scale=0.75]
    % Manager
    \node[rectangle, rounded corners, fill=primaryblue!40, draw=primaryblue, line width=2pt, minimum width=3cm, minimum height=0.9cm, font=\bfseries] (manager) {AIServiceManager};
    
    % Services
    \node[rectangle, rounded corners, fill=secondarygreen!30, draw=secondarygreen, minimum width=2.5cm, minimum height=0.8cm, below left=0.8cm and 0.5cm of manager] (qwen) {Qwen2.5 Base};
    \node[rectangle, rounded corners, fill=accentorange!30, draw=accentorange, minimum width=2.5cm, minimum height=0.8cm, below=0.8cm of manager] (whisper) {Whisper STT};
    \node[rectangle, rounded corners, fill=alertred!30, draw=alertred, minimum width=2.5cm, minimum height=0.8cm, below right=0.8cm and 0.5cm of manager] (lora) {4 LoRA Adapters};
    
    % Arrows
    \draw[-Stealth, thick] (manager) -- (qwen);
    \draw[-Stealth, thick] (manager) -- (whisper);
    \draw[-Stealth, thick] (manager) -- (lora);
\end{tikzpicture}
\end{center}

\newpage

% ===== SECTION 6: DATA FLOW =====
\section{Luồng Dữ Liệu (Data Flow)}

\subsection{Gửi Tin Nhắn - Send Message Flow}

\begin{center}
\begin{tikzpicture}[
    node distance=0.6cm,
    flowstep/.style={
        rectangle,
        rounded corners=3pt,
        minimum width=2.5cm,
        minimum height=0.9cm,
        align=center,
        font=\scriptsize,
        draw=darkgray
    },
    arrow/.style={
        -Stealth,
        thick,
        darkgray
    }
]

% Steps
\node[flowstep, fill=primaryblue!20] (s1) {1. User Input};
\node[flowstep, fill=primaryblue!20, right=of s1] (s2) {2. ChatProvider};
\node[flowstep, fill=secondarygreen!20, right=of s2] (s3) {3. SendMessage\\UseCase};
\node[flowstep, fill=accentorange!20, right=of s3] (s4) {4. ChatRepo};
\node[flowstep, fill=alertred!20, below=0.8cm of s4] (s5) {5. AI Service};
\node[flowstep, fill=accentorange!20, left=of s5] (s6) {6. Save to\\SQLite};
\node[flowstep, fill=secondarygreen!20, left=of s6] (s7) {7. Return\\Response};
\node[flowstep, fill=primaryblue!20, left=of s7] (s8) {8. Update UI};

% Arrows
\draw[arrow] (s1) -- (s2);
\draw[arrow] (s2) -- (s3);
\draw[arrow] (s3) -- (s4);
\draw[arrow] (s4) -- (s5);
\draw[arrow] (s5) -- (s6);
\draw[arrow] (s6) -- (s7);
\draw[arrow] (s7) -- (s8);

\end{tikzpicture}
\end{center}

\subsection{Chi Tiết Sequence Diagram}

\begin{center}
\begin{tikzpicture}[
    actor/.style={
        rectangle,
        rounded corners=3pt,
        minimum width=2cm,
        minimum height=0.8cm,
        text centered,
        font=\small\bfseries,
        fill=#1!30,
        draw=#1
    }
]

% Actors
\node[actor=primaryblue] (user) at (0,0) {User};
\node[actor=primaryblue] (provider) at (3,0) {Provider};
\node[actor=secondarygreen] (usecase) at (6,0) {UseCase};
\node[actor=accentorange] (repo) at (9,0) {Repository};
\node[actor=alertred] (api) at (12,0) {AI API};

% Lifelines
\draw[dashed, gray] (user) -- (0,-8);
\draw[dashed, gray] (provider) -- (3,-8);
\draw[dashed, gray] (usecase) -- (6,-8);
\draw[dashed, gray] (repo) -- (9,-8);
\draw[dashed, gray] (api) -- (12,-8);

% Messages
\draw[-Stealth, thick] (0,-1) -- node[above, font=\tiny] {sendMessage(text)} (3,-1.5);
\draw[-Stealth, thick] (3,-1.8) -- node[above, font=\tiny] {execute()} (6,-2.3);
\draw[-Stealth, thick] (6,-2.6) -- node[above, font=\tiny] {sendToAI()} (9,-3.1);
\draw[-Stealth, thick] (9,-3.4) -- node[above, font=\tiny] {POST /generate} (12,-3.9);

\draw[-Stealth, thick, dashed] (12,-4.5) -- node[above, font=\tiny] {AI Response} (9,-5);
\draw[-Stealth, thick, dashed] (9,-5.3) -- node[above, font=\tiny] {ChatMessage} (6,-5.8);
\draw[-Stealth, thick, dashed] (6,-6.1) -- node[above, font=\tiny] {Result} (3,-6.6);
\draw[-Stealth, thick, dashed] (3,-6.9) -- node[above, font=\tiny] {notifyListeners()} (0,-7.4);

\end{tikzpicture}
\end{center}

\newpage

% ===== SECTION 7: DEPENDENCY INJECTION =====
\section{Dependency Injection}

\subsection{GetIt Service Locator}

Dự án sử dụng \textbf{GetIt} làm Service Locator để quản lý dependencies:

\begin{tcolorbox}[colback=codebg, colframe=darkgray, title=injection\_container.dart]
\begin{verbatim}
final sl = GetIt.instance;

Future<void> initializeDependencies() async {
  // Core Services
  sl.registerLazySingleton<FirestoreService>(
    () => FirestoreService.instance
  );
  
  // DataSources
  sl.registerLazySingleton<ChatRemoteDataSource>(
    () => ChatRemoteDataSource(modelPath: qwenModelPath)
  );
  
  // Repositories
  sl.registerLazySingleton<ChatRepository>(
    () => ChatRepositoryImpl(remote: sl(), local: sl())
  );
  
  // UseCases
  sl.registerLazySingleton(
    () => SendMessageUseCase(sl())
  );
  
  // Providers
  sl.registerFactory(
    () => ChatProvider(sendMessage: sl(), getHistory: sl())
  );
}
\end{verbatim}
\end{tcolorbox}

\subsection{Dependency Graph}

\begin{center}
\begin{tikzpicture}[
    node distance=1cm,
    dep/.style={
        rectangle,
        rounded corners=3pt,
        minimum width=2.8cm,
        minimum height=0.8cm,
        text centered,
        font=\scriptsize,
        fill=#1!20,
        draw=#1
    }
]

% Top level
\node[dep=primaryblue] (provider) {ChatProvider};

% Use cases
\node[dep=secondarygreen, below left=1cm and 0.2cm of provider] (senduc) {SendMessageUseCase};
\node[dep=secondarygreen, below right=1cm and 0.2cm of provider] (getuc) {GetHistoryUseCase};

% Repository
\node[dep=accentorange, below=2cm of provider] (repo) {ChatRepository};

% DataSources
\node[dep=alertred, below left=1cm and 0.2cm of repo] (local) {LocalDataSource};
\node[dep=alertred, below right=1cm and 0.2cm of repo] (remote) {RemoteDataSource};

% Arrows
\draw[-Stealth] (provider) -- (senduc);
\draw[-Stealth] (provider) -- (getuc);
\draw[-Stealth] (senduc) -- (repo);
\draw[-Stealth] (getuc) -- (repo);
\draw[-Stealth] (repo) -- (local);
\draw[-Stealth] (repo) -- (remote);

\end{tikzpicture}
\end{center}

\newpage

% ===== SECTION 8: AI INTEGRATION =====
\section{Tích Hợp AI Models}

\subsection{Kiến Trúc AI Hybrid}

LexiLingo sử dụng kiến trúc Hybrid AI kết hợp Cloud và On-device models:

\begin{center}
\begin{tikzpicture}[node distance=0.6cm, scale=0.75]
    % Input
    \node[rectangle, rounded corners, fill=gray!20, draw=gray, minimum width=2cm, minimum height=0.7cm] (input) {User Input};
    
    % Router
    \node[diamond, fill=accentorange!30, draw=accentorange, minimum width=1.5cm, minimum height=1.5cm, font=\small\bfseries, below=0.8cm of input] (router) {Router};
    
    % Services
    \node[ellipse, fill=primaryblue!20, draw=primaryblue, minimum width=2cm, minimum height=0.8cm, below left=1cm and 0.5cm of router] (qwen) {Qwen Base};
    \node[rectangle, rounded corners, fill=secondarygreen!20, draw=secondarygreen, minimum width=2cm, minimum height=0.8cm, below right=1cm and 0.5cm of router] (device) {4 LoRA Adapters};
    
    % Output
    \node[rectangle, rounded corners, fill=gray!20, draw=gray, minimum width=2cm, minimum height=0.7cm, below=1.5cm of router] (output) {AI Response};
    
    % Arrows
    \draw[-Stealth, thick] (input) -- (router);
    \draw[-Stealth, thick] (router) -- (qwen);
    \draw[-Stealth, thick] (router) -- (device);
    \draw[-Stealth, thick] (qwen) -- (output);
    \draw[-Stealth, thick] (device) -- (output);
\end{tikzpicture}
\end{center}

\subsection{LoRA Adapters (Fine-tuned Models)}

Hệ thống sử dụng 4 LoRA adapters được fine-tune từ Qwen2.5-1.5B:

\begin{table}[h]
\centering
\begin{tabular}{|l|l|l|l|}
\hline
\textbf{Adapter} & \textbf{Task} & \textbf{Size} & \textbf{Loss} \\
\hline
Grammar & Sửa lỗi ngữ pháp & 151 MB & 0.77 \\
Vocabulary & Giải thích từ vựng & 151 MB & 0.71 \\
Dialogue & Hội thoại & 151 MB & 1.89 \\
Fluency & Đánh giá độ trôi chảy & 151 MB & 1.78 \\
\hline
\end{tabular}
\caption{LoRA Adapters Configuration}
\end{table}

\newpage

% ===== SECTION 9: FEATURES OVERVIEW =====
\section{Tổng Quan Các Feature Modules}

\begin{center}
\begin{tikzpicture}[
    module/.style={
        rectangle,
        rounded corners=8pt,
        minimum width=3.5cm,
        minimum height=1.5cm,
        text centered,
        font=\small\bfseries,
        drop shadow
    }
]

% Core
\node[module, fill=darkgray!30, draw=darkgray] (core) at (0,0) {\faHeart\ Core};

% Features around core
\node[module, fill=primaryblue!30, draw=primaryblue] (auth) at (-5,2) {\faUser\ Auth};
\node[module, fill=secondarygreen!30, draw=secondarygreen] (chat) at (-3,3.5) {\faComments\ Chat};
\node[module, fill=accentorange!30, draw=accentorange] (course) at (0,4) {\faBook\ Course};
\node[module, fill=alertred!30, draw=alertred] (vocab) at (3,3.5) {\faLanguage\ Vocabulary};
\node[module, fill=purple!30, draw=purple] (home) at (5,2) {\faHome\ Home};
\node[module, fill=cyan!30, draw=cyan] (user) at (5,-1) {\faUserCog\ User};
\node[module, fill=pink!30, draw=pink] (notify) at (3,-2.5) {\faBell\ Notifications};
\node[module, fill=lime!30, draw=lime] (profile) at (-3,-2.5) {\faIdCard\ Profile};

% Connections
\draw[gray, thick] (core) -- (auth);
\draw[gray, thick] (core) -- (chat);
\draw[gray, thick] (core) -- (course);
\draw[gray, thick] (core) -- (vocab);
\draw[gray, thick] (core) -- (home);
\draw[gray, thick] (core) -- (user);
\draw[gray, thick] (core) -- (notify);
\draw[gray, thick] (core) -- (profile);

\end{tikzpicture}
\end{center}

\subsection{Mô Tả Các Module}

\begin{table}[h]
\centering
\begin{tabular}{|l|p{8cm}|l|}
\hline
\textbf{Module} & \textbf{Mô Tả} & \textbf{Trạng Thái} \\
\hline
Auth & Xác thực với Google Sign-In, Firebase Auth & \textcolor{secondarygreen}{Done} \\
Chat & AI Tutor conversation, grammar check & \textcolor{secondarygreen}{Done} \\
Course & Quản lý khóa học, bài học & \textcolor{secondarygreen}{Done} \\
Vocabulary & Từ vựng, flashcards, spaced repetition & \textcolor{secondarygreen}{Done} \\
Home & Dashboard, quick stats, recent activity & \textcolor{secondarygreen}{Done} \\
User & Profile, settings, preferences & \textcolor{secondarygreen}{Done} \\
Notifications & Push notifications, reminders & \textcolor{accentorange}{In Progress} \\
Profile & User stats, achievements & \textcolor{secondarygreen}{Done} \\
\hline
\end{tabular}
\caption{Feature Modules Status}
\end{table}

\newpage

% ===== SECTION 10: DEEP LEARNING ARCHITECTURE =====
\section{Kiến Trúc Deep Learning (DL)}

\subsection{Tổng Quan Pipeline Xử Lý AI}

Khi người dùng tương tác với LexiLingo, dữ liệu đi qua một \textbf{pipeline xử lý đa tầng} như sau:

\begin{center}
\begin{tikzpicture}[
    node distance=0.5cm,
    box/.style={rectangle, rounded corners, minimum width=2.5cm, minimum height=0.7cm, align=center},
    smallbox/.style={rectangle, rounded corners, minimum width=1.8cm, minimum height=0.5cm, font=\tiny, align=center}
]
    % Input
    \node[box, fill=gray!30, draw=gray] (input) {Input\\(Voice/Text)};
    
    % STT
    \node[box, fill=primaryblue!30, draw=primaryblue, below=0.5cm of input] (stt) {STT:\\Whisper v3};
    
    % NLP Engine container
    \node[box, fill=accentorange!40, draw=accentorange, line width=2pt, minimum width=4.5cm, minimum height=2.2cm, below=0.5cm of stt] (nlp) {};
    \node[above=0.05cm, font=\small\bfseries] at (nlp.north) {NLP Engine: Qwen2.5 + LoRA};
    
    % 4 tasks inside NLP
    \node[smallbox, fill=primaryblue!20, draw=primaryblue] at ([yshift=0.35cm, xshift=-1.1cm]nlp.center) {Fluency};
    \node[smallbox, fill=secondarygreen!20, draw=secondarygreen] at ([yshift=0.35cm, xshift=1.1cm]nlp.center) {Grammar};
    \node[smallbox, fill=alertred!20, draw=alertred] at ([yshift=-0.35cm, xshift=-1.1cm]nlp.center) {Vocabulary};
    \node[smallbox, fill=purple!20, draw=purple] at ([yshift=-0.35cm, xshift=1.1cm]nlp.center) {Dialogue};
    
    % Pronunciation (parallel)
    \node[box, fill=cyan!30, draw=cyan, right=1.2cm of nlp] (pron) {HuBERT\\(Pronunciation)};
    
    % Response Aggregator
    \node[box, fill=secondarygreen!30, draw=secondarygreen, minimum width=3cm, below=0.5cm of nlp] (agg) {Response\\Aggregator};
    
    % TTS
    \node[box, fill=alertred!30, draw=alertred, below=0.5cm of agg] (tts) {TTS:\\Piper/Native};
    
    % Output
    \node[box, fill=gray!30, draw=gray, below=0.5cm of tts] (output) {Output\\(Voice/Text)};
    
    % Arrows
    \draw[-Stealth, thick] (input) -- (stt);
    \draw[-Stealth, thick] (stt) -- (nlp);
    \draw[-Stealth, thick, dashed] (stt.east) -- ++(0.4,0) |- (pron);
    \draw[-Stealth, thick] (nlp) -- (agg);
    \draw[-Stealth, thick] (pron) |- (agg);
    \draw[-Stealth, thick] (agg) -- (tts);
    \draw[-Stealth, thick] (tts) -- (output);
\end{tikzpicture}
\end{center}

\subsection{Tầng 1: Speech-to-Text (STT) - Nhận Dạng Giọng Nói}

\subsubsection{Mục đích và Vị trí}
Module STT là \textbf{điểm vào đầu tiên} khi người dùng nói tiếng Anh. Nó chuyển đổi audio thành text để các module NLP có thể xử lý.

\subsubsection{Model sử dụng: OpenAI Whisper v3}

\textbf{Tại sao chọn Whisper?}
\begin{itemize}
    \item Pre-trained trên \textbf{680,000 giờ} dữ liệu đa ngôn ngữ
    \item Hỗ trợ tốt cho \textbf{non-native speakers} (Vietnamese accent)
    \item Cung cấp \textbf{word-level timestamps} cho phân tích phát âm
    \item Open-source, có thể chạy offline hoàn toàn
\end{itemize}

\begin{table}[h]
\centering
\begin{tabular}{|l|c|c|c|c|}
\hline
\textbf{Variant} & \textbf{Params} & \textbf{WER} & \textbf{Latency} & \textbf{Use Case} \\
\hline
Whisper Large v3 & 1.5B & 3-5\% & 200-300ms & Development \\
Whisper Medium & 769M & 5-7\% & 100-200ms & High-end mobile \\
Whisper Small & 244M & 8-10\% & 50-100ms & \textbf{Production mobile} \\
\hline
\end{tabular}
\caption{Whisper Model Variants}
\end{table}

\subsubsection{Quy trình xử lý chi tiết}

\begin{tcolorbox}[colback=codebg, colframe=primaryblue, title=STT Processing Pipeline]
\begin{verbatim}
Bước 1: Audio Preprocessing
├── Resample → 16kHz mono (Whisper requirement)
├── Normalize → Peak -3dB (consistent volume)
├── VAD (Silero) → Loại bỏ silence segments
└── Chunking → 30s segments với 1s overlap

Bước 2: Whisper Inference
├── Encoder: Mel spectrogram → 1500 frames
├── Decoder: Auto-regressive text generation
├── Beam search: width=5 for accuracy
└── Output: Text + confidence per word

Bước 3: Post-processing
├── Normalize text (lowercase, punctuation)
├── Filter low-confidence words (< 0.6)
└── Extract timestamps for pronunciation

Output Example:
{
  "text": "I like learning English",
  "confidence": 0.94,
  "words": [
    {"word": "I", "start": 0.0, "end": 0.2, "conf": 0.98},
    {"word": "like", "start": 0.3, "end": 0.6, "conf": 0.95}
  ]
}
\end{verbatim}
\end{tcolorbox}

\newpage

\subsection{Tầng 2: NLP Processing Engine - Xử Lý Ngôn Ngữ}

Đây là \textbf{trái tim của hệ thống}, sử dụng kiến trúc \textbf{Unified Multi-Task} với 1 base model + 4 LoRA adapters.

\subsubsection{Tại sao chọn kiến trúc này?}

\begin{featurebox}
\textbf{So sánh: 4 models riêng vs 1 base + 4 adapters}

\textbf{4 Models riêng lẻ:}
\begin{itemize}
    \item RAM: 4 × 2GB = 8GB
    \item Storage: 4 × 900MB = 3.6GB
    \item Switching time: 2-5 giây (reload model)
\end{itemize}

\textbf{1 Base + 4 LoRA Adapters (Our Choice):}
\begin{itemize}
    \item RAM: 2GB (base) + 100MB (adapters) = \textbf{2.1GB}
    \item Storage: 900MB + 4×25MB = \textbf{1GB}
    \item Switching time: \textbf{< 1ms} (chỉ swap adapter weights)
\end{itemize}

$\Rightarrow$ \textbf{Tiết kiệm 72\% RAM, 72\% storage}
\end{featurebox}

\subsubsection{Base Model: Qwen2.5-1.5B-Instruct}

\textbf{Tại sao chọn Qwen2.5?}
\begin{itemize}
    \item \textbf{Instruction-tuned}: Hiểu trực tiếp các câu lệnh như "Check grammar", "Rate fluency"
    \item \textbf{Multilingual}: Pre-trained trên 18T tokens, mạnh về English
    \item \textbf{Efficient}: 1.5B params nhưng hiệu năng ngang GPT-3.5 trên nhiều tasks
    \item \textbf{Open-source}: Apache 2.0 license, không phụ thuộc API
\end{itemize}

\begin{table}[h]
\centering
\begin{tabular}{|l|c|c|}
\hline
\textbf{Specification} & \textbf{Development (1.5B)} & \textbf{Production (0.5B)} \\
\hline
Architecture & Decoder-only Transformer & Same \\
Layers & 28 & 24 \\
Hidden Size & 1536 & 896 \\
Attention Heads & 12 & 14 \\
Context Window & 32,768 tokens & 32,768 tokens \\
Vocab Size & 151,936 (BPE) & 151,936 \\
Model Size (Q4) & 900MB & 300MB \\
Inference Time & 100ms/sentence & 50ms/sentence \\
\hline
\end{tabular}
\caption{Qwen2.5 Model Architecture}
\end{table}

\subsubsection{LoRA (Low-Rank Adaptation) - Kỹ thuật Fine-tuning}

\textbf{LoRA hoạt động như thế nào?}

Thay vì fine-tune toàn bộ 1.5B parameters, LoRA chỉ train \textbf{low-rank matrices} được inject vào attention layers:

$$W_{new} = W_{base} + \Delta W = W_{base} + BA$$

Trong đó: $B \in \mathbb{R}^{d \times r}$, $A \in \mathbb{R}^{r \times k}$, với $r \ll \min(d, k)$

\begin{table}[h]
\centering
\begin{tabular}{|l|c|l|}
\hline
\textbf{Parameter} & \textbf{Value} & \textbf{Giải thích} \\
\hline
Rank (r) & 32 & Số chiều của low-rank decomposition \\
Alpha ($\alpha$) & 64 & Scaling factor: $\alpha/r = 2$ \\
Target Modules & 7 & q, k, v, o, gate, up, down\_proj \\
Trainable Params & 25M & Chỉ \textbf{1.7\%} của 1.5B base \\
Dropout & 0.05 & Regularization \\
\hline
\end{tabular}
\caption{LoRA Configuration}
\end{table}

\newpage

\subsection{Chi Tiết 4 LoRA Adapters}

\subsubsection{Adapter 1: Fluency Scoring - Đánh Giá Độ Trôi Chảy}

\textbf{Vị trí trong pipeline:} Được gọi \textbf{đầu tiên} sau khi nhận text từ STT để đánh giá tổng quan chất lượng câu nói.

\textbf{Nhiệm vụ:} Cho điểm từ 0.0 đến 1.0 dựa trên:
\begin{itemize}
    \item Độ tự nhiên của câu (naturalness)
    \item Độ phức tạp ngữ pháp (grammatical complexity)
    \item Sự mạch lạc (coherence)
    \item Phù hợp ngữ cảnh (contextual appropriateness)
\end{itemize}

\begin{tcolorbox}[colback=codebg, colframe=primaryblue, title=Fluency Adapter - Quy Trình]
\begin{verbatim}
INPUT FORMAT (Instruction-tuning):
<|im_start|>user
Rate the fluency of this English sentence from 0.0 to 1.0:
"Yesterday I go to library for study English"
Provide score and brief reasoning.
<|im_end|>
<|im_start|>assistant

MODEL PROCESSING:
1. Tokenize input → 45 tokens
2. Load fluency adapter weights (~25MB)
3. Forward pass through Qwen2.5 + LoRA
4. Generate score + reasoning

OUTPUT:
{
  "fluency_score": 0.45,
  "reasoning": "Multiple grammar errors: verb tense 
               (go→went), missing article (the library),
               preposition (for→to). Sentence is 
               understandable but not fluent.",
  "confidence": 0.88
}

TRAINING DATA: 1,500 samples from EFCAMDAT corpus
LOSS: MSE(predicted_score, human_score)
METRICS: MAE=0.12, Pearson r=0.91
\end{verbatim}
\end{tcolorbox}

\subsubsection{Adapter 2: Grammar Correction - Sửa Lỗi Ngữ Pháp}

\textbf{Vị trí trong pipeline:} Chạy \textbf{song song} với Fluency, sử dụng \textbf{2-tier architecture}:
\begin{enumerate}
    \item \textbf{Tier 1 - ERRANT}: Rule-based detection (nhanh, <5ms)
    \item \textbf{Tier 2 - Qwen + LoRA}: Deep correction + explanation
\end{enumerate}

\textbf{Tại sao cần 2 tiers?}
\begin{itemize}
    \item ERRANT phát hiện lỗi nhanh, chính xác cho lỗi đơn giản
    \item LLM xử lý lỗi phức tạp, cung cấp giải thích chi tiết
    \item Kết hợp: Nhanh + Chính xác + Có giải thích
\end{itemize}

\begin{tcolorbox}[colback=codebg, colframe=secondarygreen, title=Grammar Adapter - Two-Tier Pipeline]
\begin{verbatim}
TIER 1: ERRANT (Rule-based, <5ms)
Input: "Yesterday I go to library"
Output: [
  {type: "VERB:TENSE", orig: "go", corr: "went", pos: 2},
  {type: "DET", orig: "", corr: "the", pos: 4}
]

TIER 2: Qwen + LoRA (Deep correction, ~150ms)
Input: Original text + ERRANT hints
Output:
{
  "original": "Yesterday I go to library",
  "corrected": "Yesterday I went to the library",
  "errors": [
    {
      "type": "VERB_TENSE",
      "original": "go",
      "corrected": "went", 
      "explanation": "Use past tense 'went' with time 
                     marker 'yesterday'. Simple past
                     indicates completed action."
    },
    {
      "type": "MISSING_ARTICLE",
      "original": "library",
      "corrected": "the library",
      "explanation": "Use definite article 'the' for 
                     specific places known to both
                     speaker and listener."
    }
  ]
}

TRAINING DATA: 9,200 samples (BEA-2019, CoNLL-2014, FCE)
LOSS: CrossEntropy + Confidence score
METRICS: F0.5=68, Precision=78%, Recall=68%
\end{verbatim}
\end{tcolorbox}

\subsubsection{Adapter 3: Vocabulary Classification - Phân Loại Từ Vựng}

\textbf{Vị trí trong pipeline:} Chạy \textbf{song song} để xác định trình độ CEFR của người học, giúp Dialogue adapter điều chỉnh response phù hợp.

\textbf{CEFR Levels được hỗ trợ:}
\begin{itemize}
    \item \textbf{A2 (Elementary)}: Từ vựng cơ bản - go, like, friend, school
    \item \textbf{B1 (Intermediate)}: Từ vựng trung cấp - discuss, environment, improve
    \item \textbf{B2 (Upper-Intermediate)}: Từ vựng nâng cao - comprehensive, demonstrate
\end{itemize}

\begin{tcolorbox}[colback=codebg, colframe=alertred, title=Vocabulary Adapter - Classification]
\begin{verbatim}
INPUT:
"I want to discuss the environmental problems 
 that affect our community"

PROCESSING:
1. Word-level CEFR lookup (Trie-based, O(n)):
   - discuss → B1
   - environmental → B2  
   - problems → A2
   - affect → B1
   - community → B1

2. LLM refinement (context-aware):
   - Sentence structure complexity → B1-B2
   - Topic sophistication → B1

OUTPUT:
{
  "overall_level": "B1",
  "distribution": {"A2": 0.15, "B1": 0.55, "B2": 0.30},
  "key_words": {
    "B2": ["environmental"],
    "B1": ["discuss", "affect", "community"],
    "A2": ["want", "problems"]
  },
  "recommendation": "User demonstrates solid B1 level
                    with some B2 vocabulary. Ready for
                    more complex topics."
}

TRAINING DATA: 2,500 samples (Oxford Graded Readers)
LOSS: CrossEntropy with class weights [0.9, 1.0, 1.1]
METRICS: Accuracy=90%, Macro F1=0.89
\end{verbatim}
\end{tcolorbox}

\subsubsection{Adapter 4: Dialogue Response - Sinh Phản Hồi AI Tutor}

\textbf{Vị trí trong pipeline:} Chạy \textbf{cuối cùng}, tổng hợp kết quả từ 3 adapters trước để sinh response phù hợp.

\textbf{Tại sao cần AutoTutor Dialogue Corpus?}
\begin{itemize}
    \item \textbf{Socratic Tutoring}: Dạy qua câu hỏi gợi mở thay vì đưa đáp án trực tiếp
    \item \textbf{Scaffolding}: Hỗ trợ từng bước, điều chỉnh theo trình độ học sinh
    \item \textbf{Pedagogical Strategies}: Khuyến khích, phản hồi xây dựng, duy trì động lực
    \item \textbf{Conversational Coherence}: Duy trì맥络 hội thoại tự nhiên, không máy móc
\end{itemize}

\textbf{Context Variables được sử dụng:}
\begin{itemize}
    \item Fluency score từ Adapter 1
    \item Grammar errors từ Adapter 2
    \item CEFR level từ Adapter 3
    \item Conversation history (3 turns gần nhất)
\end{itemize}

\begin{tcolorbox}[colback=codebg, colframe=purple, title=Dialogue Adapter - Response Generation]
\begin{verbatim}
CONTEXT INPUT:
{
  "user_input": "Yesterday I go to library",
  "fluency_score": 0.45,
  "cefr_level": "A2",
  "errors": ["VERB_TENSE", "MISSING_ARTICLE"],
  "history": ["Hi, how are you?", "I'm fine, thanks!"]
}

PERSONA: Friendly, encouraging English tutor
TONE ADAPTATION:
- A2: Simple words, short sentences, more encouragement
- B1: Natural conversation, moderate complexity
- B2: Near-native interaction, subtle corrections

GENERATION CONFIG:
- Temperature: 0.7 (balanced creativity)
- Top-p: 0.9 (nucleus sampling)
- Max tokens: 100
- Stop tokens: ["<|im_end|>"]

OUTPUT:
{
  "response": "Good effort! 👍 You should say 'Yesterday 
              I went to the library.' Remember: use 
              'went' for past actions, and 'the' before 
              'library'. What did you do there?",
  "corrections_included": true,
  "follow_up_question": true,
  "tone": "encouraging",
  "adapted_for_level": "A2"
}

TRAINING DATA: 5,200 samples
├── AutoTutor Dialogue Corpus: 1,800 samples
│   (Tutorial dialogues, Socratic questioning)
├── Intel/orca-dpo-pairs: 1,500 samples
│   (High-quality instruction following)
├── Custom tutoring conversations: 1,900 samples
│   (English learning scenarios)
└── Total: 5,200 tutoring-style interactions

LOSS: CrossEntropy on response tokens
METRICS: Quality=96%, Appropriateness=94%
Tutoring Style Score: 89% (human evaluation)
\end{verbatim}
\end{tcolorbox}

\newpage

\subsection{Tầng 3: Pronunciation Analysis - Phân Tích Phát Âm}

\subsubsection{Model: HuBERT-large (facebook/hubert-large-ls960)}

\textbf{Vị trí:} Chạy \textbf{song song} với NLP Engine, nhận audio trực tiếp từ STT.

\textbf{HuBERT (Hidden-Unit BERT) hoạt động như thế nào?}
\begin{enumerate}
    \item \textbf{Self-supervised pre-training}: Học representations từ 960h LibriSpeech
    \item \textbf{Phoneme recognition}: CTC decoding để nhận dạng 44 ARPAbet phonemes
    \item \textbf{Forced alignment}: So sánh với native speaker reference
\end{enumerate}

\begin{tcolorbox}[colback=codebg, colframe=cyan, title=Pronunciation Analysis Pipeline]
\begin{verbatim}
STEP 1: Phoneme Recognition (HuBERT + CTC)
Audio: "think" → /θ/ /ɪ/ /ŋ/ /k/
User pronounced: /s/ /ɪ/ /ŋ/ /k/

STEP 2: Forced Alignment (DTW Algorithm)
Compare user phonemes vs native reference:
| Position | Expected | User    | Status      |
|----------|----------|---------|-------------|
| 1        | /θ/      | /s/     | SUBSTITUTION|
| 2        | /ɪ/      | /ɪ/     | CORRECT     |
| 3        | /ŋ/      | /ŋ/     | CORRECT     |
| 4        | /k/      | /k/     | CORRECT     |

STEP 3: Error Analysis
{
  "phoneme_accuracy": 0.75,
  "errors": [
    {
      "type": "SUBSTITUTION",
      "expected": "/θ/ (voiceless dental fricative)",
      "actual": "/s/ (voiceless alveolar fricative)",
      "word": "think",
      "tip": "Place tongue between teeth for 'th' sound"
    }
  ],
  "prosody": {
    "stress_pattern": "correct",
    "intonation": "slightly flat"
  }
}

MODEL: facebook/hubert-large-ls960 (960M params)
LATENCY: <500ms per utterance
\end{verbatim}
\end{tcolorbox}

\subsection{Tầng 4: Text-to-Speech (TTS) - Chuyển Văn Bản Thành Giọng Nói}

\textbf{Hệ thống sử dụng 3-tier TTS} để cân bằng chất lượng và tốc độ:

\begin{table}[h]
\centering
\begin{tabular}{|l|l|l|l|l|}
\hline
\textbf{Tier} & \textbf{Model} & \textbf{Size} & \textbf{Latency} & \textbf{Use Case} \\
\hline
1 & Native OS TTS & 0MB & <100ms & Quick feedback \\
2 & Piper TTS (VITS) & 30-60MB & 100-300ms & Pronunciation demos \\
3 & Cloud TTS & 0MB & 300-800ms & Premium quality \\
\hline
\end{tabular}
\caption{TTS Tier System}
\end{table}

\subsection{Training Pipeline Chi Tiết}

\begin{tcolorbox}[colback=codebg, colframe=darkgray, title=LoRA Training Pipeline]
\begin{verbatim}
PHASE 1: Data Preparation
├── Collect domain-specific data:
│   • Fluency: 1,500 (EFCAMDAT)
│   • Grammar: 9,200 (BEA-2019, CoNLL-2014, FCE)
│   • Vocabulary: 2,500 (Oxford Graded Readers)
│   • Dialogue: 5,200 (AutoTutor, Intel/orca, Custom)
├── Format: Instruction-tuning template (<|im_start|>...)
├── Split: 70% train / 15% val / 15% test
└── Augmentation: Back-translation, paraphrase

PHASE 2: Environment Setup
├── Hardware: Mac M1/M2 32GB or NVIDIA GPU
├── Framework: PyTorch + Transformers + PEFT
├── Precision: BFloat16 (faster on Apple Silicon)
└── Batch: 8 (gradient accumulation 4 → effective 32)

PHASE 3: Training
├── Optimizer: AdamW (lr=2e-4, weight_decay=0.01)
├── Scheduler: Cosine with 3% warmup
├── Epochs: 5-7
├── Gradient clipping: 1.0
└── Early stopping: patience=3

PHASE 4: Evaluation
├── Fluency: MAE, Pearson correlation
├── Grammar: F0.5, Precision, Recall
├── Vocabulary: Accuracy, Macro F1
└── Dialogue: Human evaluation (quality, appropriateness)

PHASE 5: Production Optimization
├── Quantization: INT8 / INT4
├── Knowledge distillation: 1.5B → 0.5B
└── Export: ONNX / TensorRT for mobile
\end{verbatim}
\end{tcolorbox}

\subsection{So Sánh Development vs Production}

\begin{table}[h]
\centering
\begin{tabular}{|l|c|c|}
\hline
\textbf{Component} & \textbf{Development} & \textbf{Production} \\
\hline
\multicolumn{3}{|c|}{\textbf{Speech-to-Text}} \\
\hline
Model & Whisper Large v3 & Whisper Small \\
Size & 1.5GB & 500MB \\
WER & 3-5\% & 8-10\% \\
RAM & 4GB & 1.5GB \\
\hline
\multicolumn{3}{|c|}{\textbf{NLP Engine (Unified)}} \\
\hline
Base Model & Qwen2.5-1.5B & Qwen2.5-0.5B \\
LoRA Adapters & 4 × 25MB & 4 × 8MB \\
RAM & 2GB & 600MB \\
Quality & 96\% & 91\% \\
\hline
\multicolumn{3}{|c|}{\textbf{Pronunciation}} \\
\hline
Model & HuBERT-large & Server-side \\
Params & 960M & API call \\
\hline
\multicolumn{3}{|c|}{\textbf{TTS}} \\
\hline
Model & Native + Piper & Native only \\
Size & 0-60MB & 0MB \\
\hline
\multicolumn{3}{|c|}{\textbf{TOTAL}} \\
\hline
RAM & 6-7GB & 2.4GB \\
Storage & 3GB & 1GB \\
Latency & \textasciitilde 600ms & \textasciitilde 400ms \\
\hline
\end{tabular}
\caption{Development vs Production Comparison}
\end{table}

\newpage

% ===== SECTION 11: SO SÁNH VỚI CÁC NGHIÊN CỨU KHÁC =====
\section{So Sánh Với Các Nghiên Cứu và Model Khác}

\subsection{Tổng Quan Các Hướng Tiếp Cận}

Trong lĩnh vực học tiếng Anh có sự hỗ trợ của AI, đã có nhiều nghiên cứu và hệ thống được phát triển với các phương pháp khác nhau. Phần này so sánh kiến trúc LexiLingo với các nghiên cứu tiêu biểu đã được công bố.

\begin{table}[h]
\centering
\small
\begin{tabular}{|l|c|c|c|c|}
\hline
\textbf{Khía Cạnh} & \textbf{LexiLingo} & \textbf{Baseline 1} & \textbf{Baseline 2} & \textbf{Cloud-based} \\
\hline
Architecture & Hybrid & Rule-based & Pure ML & Cloud API \\
On-device & \checkmark & \checkmark & $\times$ & $\times$ \\
Offline Support & \checkmark & \checkmark & $\times$ & $\times$ \\
Multi-task & \checkmark & $\times$ & Limited & \checkmark \\
Personalized & \checkmark & $\times$ & Limited & \checkmark \\
Cost & Low & Very Low & Low & High \\
\hline
\end{tabular}
\caption{So Sánh Các Hướng Tiếp Cận}
\end{table}

\subsection{Grammatical Error Correction (GEC)}

\subsubsection{So Sánh Với Các Nghiên Cứu Hiện Đại}

\textbf{1. Byte-Level vs Subword Approach (Ingólfsdóttir et al., 2023)}

Nghiên cứu của Ingólfsdóttir et al. cho ngôn ngữ Iceland so sánh byte-level encoding với subword tokenization, cho thấy byte-level models hiệu quả hơn cho các ngôn ngữ morphologically rich.

\begin{itemize}
    \item \textbf{Approach}: Synthetic data generation + Fine-tuning
    \item \textbf{Advantage}: Xử lý tốt spelling và semantic errors
    \item \textbf{LexiLingo Comparison}: Chúng ta sử dụng subword BPE (151K vocab) vì tiếng Anh ít phức tạp morphology hơn Iceland
\end{itemize}

\textbf{2. State-of-the-Art GEC (Omelianchuk et al., 2024)}

Nghiên cứu "Pillars of Grammatical Error Correction" đạt SOTA với F\textsubscript{0.5} = 72.8 (CoNLL-2014) và 81.4 (BEA).

\begin{table}[h]
\centering
\begin{tabular}{|l|c|c|c|}
\hline
\textbf{System} & \textbf{CoNLL-2014} & \textbf{BEA-2019} & \textbf{Approach} \\
\hline
Omelianchuk et al. & 72.8 & 81.4 & Ensemble + LLM \\
gT5 (Rothe et al.) & 69.2 & 78.6 & T5-11B \\
\textbf{LexiLingo} & 68.0* & 76.5* & Qwen2.5 1.5B + LoRA \\
\hline
\multicolumn{4}{l}{*Estimated based on similar training data and model size}
\end{tabular}
\caption{GEC Performance Comparison (F\textsubscript{0.5} Score)}
\end{table}

\textbf{Trade-offs của LexiLingo:}
\begin{itemize}
    \item \textbf{Ưu điểm}: Model size nhỏ hơn 5-10x (1.5B vs 11B), chạy được on-device
    \item \textbf{Nhược điểm}: Accuracy thấp hơn \textasciitilde 5-7\%, nhưng chấp nhận được cho mobile app
    \item \textbf{Solution}: Two-tier approach (ERRANT + Qwen) để cân bằng speed và accuracy
\end{itemize}

\textbf{3. Low-Resource GEC (Keita et al., 2024)}

Nghiên cứu về Zarma language so sánh rule-based, MT models (M2M100), và LLMs (mT5).

\begin{itemize}
    \item \textbf{Finding}: M2M100 đạt 95.82\% detection, 78.90\% correction accuracy
    \item \textbf{LexiLingo Alignment}: Chúng ta cũng sử dụng MT approach nhưng với Qwen thay vì M2M vì:
    \begin{itemize}
        \item Qwen có instruction-following tốt hơn
        \item Hỗ trợ multi-task (grammar + dialogue + vocabulary)
        \item Community support lớn hơn cho fine-tuning
    \end{itemize}
\end{itemize}

\textbf{4. Explainable GEC (Kaneko \& Okazaki, 2023)}

Controlled Generation with Prompt Insertion (PI) giúp LLMs giải thích lý do sửa lỗi.

\begin{itemize}
    \item \textbf{Innovation}: Tự động extract correction points và insert vào prompt
    \item \textbf{LexiLingo Integration}: Chúng ta áp dụng tương tự trong Dialogue adapter:
\end{itemize}

\begin{tcolorbox}[colback=codebg, colframe=secondarygreen, title=Explainable Correction Example]
\begin{verbatim}
INPUT: "Yesterday I go to library"
CORRECTION POINTS: [("go" → "went", VERB:TENSE)]

DIALOGUE RESPONSE WITH EXPLANATION:
"Good try! You used 'go' but since you're talking 
about yesterday (past time), you need the past tense 
'went'. Try: 'Yesterday I went to the library.'"
\end{verbatim}
\end{tcolorbox}

\newpage

\subsection{Pronunciation Assessment}

\subsubsection{Datasets và Benchmarks}

\textbf{1. SpeechOcean762 (Zhang et al., 2021)}

Corpus gồm 5,000 utterances từ 250 non-native speakers với annotations ở sentence, word, và phoneme level.

\begin{table}[h]
\centering
\begin{tabular}{|l|c|c|}
\hline
\textbf{Feature} & \textbf{SpeechOcean762} & \textbf{LexiLingo Dataset} \\
\hline
Speakers & 250 & TBD (collecting) \\
Utterances & 5,000 & Target: 10,000 \\
Annotation Levels & 3 (sent/word/phone) & 2 (word/phone) \\
Expert Annotators & 5 & 3 \\
Age Groups & Adults + Children & Adults only (initial) \\
License & Open-source & Open-source \\
\hline
\end{tabular}
\caption{Pronunciation Dataset Comparison}
\end{table}

\textbf{2. HuBERT vs Alternatives}

\begin{table}[h]
\centering
\small
\begin{tabular}{|l|c|c|c|c|}
\hline
\textbf{Model} & \textbf{Size} & \textbf{Pre-training} & \textbf{Phoneme Acc} & \textbf{Use Case} \\
\hline
HuBERT-large & 960M & 960h LibriSpeech & 92\% & \textbf{LexiLingo} \\
Wav2Vec 2.0 & 317M & 960h LibriSpeech & 88\% & Alternative \\
ECAPA-TDNN & 24M & VoxCeleb & N/A & Accent class. \\
Whisper & 1.5B & 680K hours & 85\%* & STT primary \\
\hline
\multicolumn{5}{l}{*When used for pronunciation, not optimal for this task}
\end{tabular}
\caption{Pronunciation Models Comparison}
\end{table}

\textbf{Lý do chọn HuBERT:}
\begin{itemize}
    \item \textbf{Self-supervised}: Học representations tốt từ unlabeled audio
    \item \textbf{Phoneme-focused}: Pre-training target là phoneme prediction
    \item \textbf{Forced alignment}: Dễ dàng compare với native reference
    \item \textbf{Fine-tuning friendly}: Có thể adapt cho Vietnamese-accented English
\end{itemize}

\textbf{3. Multi-aspect Assessment (Do et al., 2024)}

Acoustic Feature Mixup for balanced scoring across multiple aspects:

\begin{itemize}
    \item \textbf{Innovation}: Mixup strategies để address data imbalance
    \item \textbf{LexiLingo Adoption}: Chúng ta sử dụng tương tự cho fluency scoring:
    \begin{itemize}
        \item Linear interpolation với in-batch average
        \item Goodness-of-pronunciation (GOP) features
        \item Fine-grained error-rate từ ASR comparison
    \end{itemize}
\end{itemize}

\subsection{Dialogue Systems và AI Tutoring}

\subsubsection{Neural Dialog Tutoring (Macina et al., 2023)}

Nghiên cứu phân tích challenges trong neural dialog tutoring:

\textbf{Key Findings:}
\begin{itemize}
    \item LLMs perform well trong constrained scenarios (ít concepts, rõ ràng strategies)
    \item Struggle với unconstrained scenarios và equitable tutoring
    \item 45\% conversations có model reasoning errors
\end{itemize}

\textbf{LexiLingo Approach:}
\begin{itemize}
    \item \textbf{Constrained Domain}: Focus vào English learning (specific concepts)
    \item \textbf{Explicit Strategies}: Socratic questioning, error correction, vocabulary expansion
    \item \textbf{Context-Aware}: Sử dụng fluency/grammar/vocabulary context để guide response
\end{itemize}

\begin{tcolorbox}[colback=codebg, colframe=purple, title=Constrained vs Unconstrained Tutoring]
\begin{verbatim}
UNCONSTRAINED (Challenging):
Student: "Tell me about climate change"
→ Too open-ended, many concepts, unclear goal

CONSTRAINED (LexiLingo):
Student: "Yesterday I go library"
Context: {fluency: 0.45, errors: [VERB:TENSE, DET]}
→ Clear goal: Fix grammar, improve fluency
→ Response: "Good try! Let's fix two things..."
\end{verbatim}
\end{tcolorbox}

\subsubsection{TUTORING Bot (Chae et al., 2023)}

Instruction-grounded conversational agent với multi-task learning:

\begin{table}[h]
\centering
\begin{tabular}{|l|c|c|}
\hline
\textbf{Feature} & \textbf{TUTORING Bot} & \textbf{LexiLingo} \\
\hline
Multi-task Learning & \checkmark & \checkmark \\
Instruction Following & \checkmark & \checkmark \\
Teaching Action Inference & \checkmark & \checkmark \\
Progress Monitoring & \checkmark & \checkmark \\
Domain & English learning & English learning \\
Architecture & Single LLM & 4 LoRA adapters \\
Context Integration & Instruction only & Multi-modal context \\
\hline
\end{tabular}
\caption{Tutoring System Comparison}
\end{table}

\textbf{LexiLingo Advantages:}
\begin{itemize}
    \item \textbf{Richer Context}: Integrate fluency, grammar, vocabulary signals
    \item \textbf{Modular Design}: Separate adapters cho separate concerns
    \item \textbf{Fast Switching}: < 1ms adapter swap vs reload entire model
\end{itemize}

\subsubsection{AutoTutor Dialogue Corpus}

\textbf{Why AutoTutor for Training Data?}

AutoTutor là một intelligent tutoring system với 20+ years research:

\begin{itemize}
    \item \textbf{Socratic Questioning}: Dạy qua câu hỏi gợi mở thay vì trực tiếp đưa đáp án
    \item \textbf{Scaffolding}: Hỗ trợ dần dần (fading) khi học sinh progress
    \item \textbf{Conversational Coherence}: Duy trì multi-turn dialogues tự nhiên
    \item \textbf{Proven Effectiveness}: Nhiều studies cho thấy learning gains
\end{itemize}

\begin{table}[h]
\centering
\begin{tabular}{|l|c|c|c|}
\hline
\textbf{Training Data} & \textbf{Samples} & \textbf{Focus} & \textbf{Quality} \\
\hline
AutoTutor Corpus & 1,800 & Socratic tutoring & High \\
Intel/orca-dpo-pairs & 1,500 & Instruction following & High \\
Custom conversations & 1,900 & English learning & Medium \\
\hline
\textbf{Total} & \textbf{5,200} & \textbf{Mixed} & \textbf{High} \\
\hline
\end{tabular}
\caption{Dialogue Training Data Sources}
\end{table}

\subsection{Fluency và Writing Assessment}

\subsubsection{Automated Essay Scoring (AES)}

\textbf{1. Multi-dimensional Scoring (Sun \& Wang, 2024)}

Automated essay scoring across multiple dimensions (vocabulary, grammar, coherence):

\begin{itemize}
    \item \textbf{Approach}: Fine-tuning + Multiple regression
    \item \textbf{Metrics}: Precision, F1, Quadratic Weighted Kappa
    \item \textbf{LexiLingo Similarity}: Chúng ta cũng score multiple dimensions nhưng real-time
\end{itemize}

\textbf{2. CEFR Speaking Assessment (Scaria et al., 2024)}

EvalYaks - LoRA fine-tuned models for CEFR B2 assessment:

\begin{table}[h]
\centering
\begin{tabular}{|l|c|c|c|}
\hline
\textbf{System} & \textbf{Base Model} & \textbf{Accuracy} & \textbf{Variation} \\
\hline
EvalYaks & Mistral 7B & 96\% & 0.35 levels \\
\textbf{LexiLingo} & Qwen2.5 1.5B & 90\%* & 0.4 levels* \\
Best Baseline & GPT-4 & 88\% & 0.45 levels \\
\hline
\multicolumn{4}{l}{*Estimated based on similar evaluation setup}
\end{tabular}
\caption{CEFR Assessment Performance}
\end{table}

\textbf{Key Insight}: Smaller models (1.5-7B) với high-quality CEFR-aligned data có thể outperform larger models.

\subsubsection{LLM-as-a-Judge (Son et al., 2024)}

Nghiên cứu về limitations của LLMs as evaluators:

\textbf{Findings:}
\begin{itemize}
    \item LLMs fail to detect errors trong > 50\% cases
    \item Reference-based evaluation tốt hơn single-answer/pairwise
    \item Need careful prompt design
\end{itemize}

\textbf{LexiLingo Solution:}
\begin{itemize}
    \item \textbf{Hybrid Approach}: ERRANT rules + LLM refinement
    \item \textbf{Reference-based}: So sánh với native speaker phonemes (pronunciation)
    \item \textbf{Explicit Criteria}: Clear rubrics trong prompt (fluency 0-1, grammar error types)
\end{itemize}

\newpage

\subsection{Speech Recognition cho Language Learning}

\subsubsection{Whisper vs Alternatives}

\begin{table}[h]
\centering
\small
\begin{tabular}{|l|c|c|c|c|c|}
\hline
\textbf{Model} & \textbf{WER} & \textbf{Multilingual} & \textbf{Size} & \textbf{Latency} & \textbf{Offline} \\
\hline
Whisper Large & 3-5\% & 99 langs & 1.5GB & 300ms & \checkmark \\
Whisper Small & 8-10\% & 99 langs & 500MB & 100ms & \checkmark \\
Google STT & 2-4\% & 125 langs & Cloud & 200ms & $\times$ \\
Azure STT & 3-5\% & 100 langs & Cloud & 250ms & $\times$ \\
\hline
\end{tabular}
\caption{STT Models Comparison}
\end{table}

\textbf{Tại sao chọn Whisper?}
\begin{enumerate}
    \item \textbf{Offline-first}: Crucial cho educational apps (không phụ thuộc network)
    \item \textbf{Robust}: Pre-trained trên 680K hours diverse data
    \item \textbf{Timestamps}: Cung cấp word-level timestamps cho pronunciation analysis
    \item \textbf{Open-source}: Không có cost per request, full control
\end{enumerate}

\subsubsection{Accented English Recognition}

\textbf{AESRC2020 Challenge (Shi et al., 2021)}

160 hours accented English từ 8 countries, test trên 2 unseen accents:

\begin{itemize}
    \item \textbf{Challenge}: Model generalization cho unseen accents
    \item \textbf{LexiLingo Focus}: Vietnamese-accented English
    \item \textbf{Strategy}: 
    \begin{itemize}
        \item Collect Vietnamese speaker data
        \item Fine-tune Whisper Small trên accent-specific data
        \item Use accent classification (Wav2Vec 2.0) để detect Vietnamese accent
        \item Switch to fine-tuned model khi detect Vietnamese accent
    \end{itemize}
\end{itemize}

\subsection{Kiến Trúc Unified Multi-Task}

\subsubsection{So Sánh Với Các Kiến Trúc Khác}

\begin{table}[h]
\centering
\small
\begin{tabular}{|p{3cm}|p{2.5cm}|p{2.5cm}|p{2.5cm}|}
\hline
\textbf{Architecture} & \textbf{Separate Models} & \textbf{Multi-task Single Model} & \textbf{LexiLingo (LoRA)} \\
\hline
Number of Models & 4 independent & 1 shared & 1 base + 4 adapters \\
\hline
Total Size & 8GB & 2GB & 2.1GB \\
\hline
Task Switching & 2-5s (reload) & Instant & < 1ms (swap) \\
\hline
Training & Independent & Joint training & Sequential LoRA \\
\hline
Flexibility & High & Low & High \\
\hline
Maintenance & Hard & Easy & Moderate \\
\hline
Performance & Best per task & Moderate & Near-best \\
\hline
\end{tabular}
\caption{Multi-Task Architecture Comparison}
\end{table}

\textbf{Advantages của LoRA Architecture:}
\begin{enumerate}
    \item \textbf{Modularity}: Có thể update từng adapter independently
    \item \textbf{Efficiency}: Share base model weights (1.5B params)
    \item \textbf{Scalability}: Dễ thêm adapters cho new tasks (e.g., writing style, idioms)
    \item \textbf{Performance}: Gần với separate models nhưng 4x nhỏ hơn
\end{enumerate}

\subsection{Kết Luận So Sánh}

\begin{tcolorbox}[colback=lightgray, colframe=primaryblue, title=LexiLingo Position in Research Landscape]
\textbf{Strengths:}
\begin{itemize}
    \item \textbf{Mobile-First}: Optimized cho on-device inference (1-2GB RAM)
    \item \textbf{Hybrid Intelligence}: Kết hợp rule-based + neural approaches
    \item \textbf{Multi-Modal}: Integrate speech, text, phoneme analysis
    \item \textbf{Practical Trade-offs}: 5-7\% accuracy loss cho 5-10x smaller size
\end{itemize}

\textbf{Areas for Improvement:}
\begin{itemize}
    \item Grammar correction F0.5: 68 vs SOTA 72.8 (gap: 4.8 points)
    \item Pronunciation analysis: Currently server-side, need on-device solution
    \item Dataset size: 5.2K dialogue samples vs 50K+ in commercial systems
    \item Accent adaptation: Currently Vietnamese only, need multi-accent support
\end{itemize}

\textbf{Novel Contributions:}
\begin{itemize}
    \item First Vietnamese-focused English learning app với on-device AI
    \item Two-tier GEC (ERRANT + Qwen) cho speed-accuracy balance
    \item Context-aware dialogue (fluency + grammar + vocabulary signals)
    \item Practical LoRA architecture cho resource-constrained devices
\end{itemize}
\end{tcolorbox}

\newpage

% ===== SECTION 12: CONCLUSION =====
\section{Kết Luận}

\subsection{Tóm Tắt Kiến Trúc}

\begin{infobox}[Điểm Nổi Bật]
\begin{itemize}
    \item \textbf{Clean Architecture}: Tách biệt rõ ràng Presentation - Domain - Data
    \item \textbf{Feature-First}: Tổ chức code theo feature, dễ maintain
    \item \textbf{Dependency Injection}: Sử dụng GetIt cho loose coupling
    \item \textbf{Hybrid AI}: Kết hợp Cloud + On-device models
    \item \textbf{Cross-Platform}: Flutter cho iOS, Android, Web
\end{itemize}
\end{infobox}

\subsection{Roadmap Phát Triển}

\begin{enumerate}
    \item \textbf{Phase 1 (Current)}: Core features - Chat, Course, Vocabulary
    \item \textbf{Phase 2}: Voice interaction - STT/TTS integration
    \item \textbf{Phase 3}: Advanced AI - On-device inference với LoRA adapters
    \item \textbf{Phase 4}: Social features - Leaderboard, community
\end{enumerate}

\subsection{Thông Tin Dự Án}

\begin{table}[h]
\centering
\begin{tabular}{ll}
\toprule
\textbf{Thông Tin} & \textbf{Chi Tiết} \\
\midrule
Version & v0.2.0 \\
Flutter SDK & 3.29+ \\
Dart SDK & 3.8.1+ \\
Platforms & iOS, Android, Web \\
Repository & \href{https://github.com/InfinityZero3000/LexiLingo}{github.com/InfinityZero3000/LexiLingo} \\
Branch & feature \\
\bottomrule
\end{tabular}
\end{table}

\vfill

\begin{center}
\textit{Document generated on \today}\\[0.5cm]
{\Large\color{primaryblue}\textbf{LexiLingo} - Learn English with AI}
\end{center}

\end{document}
